% CREATED BY DAVID FRISK, 2016
\chapter{Conclusion and future work}
% You may consider to instead divide this chapter into discussion of the results and a summary.
% \section{Discussion}
\section{Summary}

In this project we set out to improve the overall usability of ud2gf, focusing on issues with performance, difficulty in debugging and lack of flexibility regarding differing tree shapes.

\paragraph*{Performance}
% Resounding success
We reduced some combinatorial explosions and prevented same candidates being generated multiple times in the tree generation phase. The performance was greatly improved, with the slowest sentence in our test improving from 85 seconds to 44 milliseconds. One of the optimizations improved the performance by a linear factor depending on the depth of the tree, while the other optimization improved the performance by an exponential factor depending on the width of the tree.

We could also see that garbage collection can have a large impact on the performance of Haskell programs. However, in our case it could easily be alleviated, without a significant impact on total memory use, by increasing the initial heap size for the program.

Another future option to improve performance would be to try using the GF C runtime instead of the GF Haskell runtime.

% Future work?
% A yet unexplored option is to try using the GF C runtime instead of the GF Haskell runtime. This would also solve the garbage collection issues, since C is not a garbage collected language and the C heap is not touched by the Haskell garbage collector.
% Another alternative would be to switch to using the GF C runtime instead of the GF Haskell runtime.

\paragraph*{Debugging}
We added a debugger feature, where the user can query ud2gf for an explanation why a certain function was not applied. This feature was much less mature compared to the performance improvements, but the single user who tested it gave positive feedback.
% But the project changed its approach away from the UD-based workflow, so this tool didn't get much use after the initial experiments.

\paragraph*{Tree shape flexibility}
We made the macro system recursive, which made it possible to do conversions that require more context than just the immediate children. The improved macro system solved the issue, in that we can now express things we could not do before, but the syntax is not very user-friendly.

% \section{Conclusion}


\section{Future work}

%

The translation described by a labels file is not one-to-one and there are often many possible GF trees that a UD tree could be translated to. The possible trees are currently ranked by completeness, as in how many of the words are included in the generated tree. However this ranking is incomplete and in case two possible trees, with the same GF category, cover the same words, an arbitrary tree will be chosen. A better choice could be to also check the linearization of these trees and rank those whose linearization is more similar to the original string higher. It would also be possible to completely exclude trees with differing linearization, but that would run counter to the goal of robustness.

There are still many cases that the conversion can not handle, for example: If a GF grammar has functions of type \lstinline{: A -> A}) or some combination of functions that can achieve that type, we can get an infinite loop of that function being repeatedly applied, so it is important that such functions are not included in the labels file. This is not detected automatically, so it is up to the user to notice it and fix the issue.

The debugging tool is mostly a rough, but useful, prototype, so it would benefit from more polishing.

The expanded macro capabilities would benefit from having their own syntax to get a nicer and more user-friendly interface.

