% CREATED BY DAVID FRISK, 2016

% BASIC SETTINGS
\usepackage{moreverb}								% List settings
\usepackage{textcomp}								% Fonts, symbols etc.
\usepackage{lmodern}								% Latin modern font
\usepackage{helvet}									% Enables font switching
\usepackage[T1]{fontenc}							% Output settings
\usepackage[english]{babel}							% Language settings
\usepackage[utf8]{inputenc}							% Input settings
\usepackage{amsmath}								% Mathematical expressions (American mathematical society)
\usepackage{amssymb}								% Mathematical symbols (American mathematical society)
\usepackage[pdf]{graphviz}                          % dot-files directly in latex
\usepackage{graphicx}								% Figures
% \usepackage{subfig}									% Enables subfigures
\usepackage{caption}
\usepackage{subcaption}
\usepackage{acronym}                                % For a list of abbreviations
\usepackage{glossaries}
\usepackage{tikz-dependency}
% \usepackage{tikzmark}
% \usepackage{tikz}
% \usetikzlibrary{external}
% \tikzexternalize[prefix=tikz/]
\usetikzlibrary{decorations.pathmorphing}

\numberwithin{equation}{chapter}					% Numbering order for equations
\numberwithin{figure}{chapter}						% Numbering order for figures
\numberwithin{table}{chapter}						% Numbering order for tables
\usepackage{minted}					% Enables source code listings
\usepackage{chemfig}								% Chemical structures
\usepackage[top=3cm, bottom=3cm,
			inner=3cm, outer=3cm]{geometry}			% Page margin lengths
\usepackage{eso-pic}								% Create cover page background
\newcommand{\backgroundpic}[3]{
	\put(#1,#2){
	\parbox[b][\paperheight]{\paperwidth}{
	\centering
	\includegraphics[width=\paperwidth,height=\paperheight,keepaspectratio]{#3}}}}
\usepackage{float} 									% Enables object position enforcement using [H]
\usepackage{parskip}								% Enables vertical spaces correctly
\usepackage{datetime2} % date formatting tools - ISO-date YYYY-MM-DD
\usepackage{microtype} % Microtypography - improves readability and appearance of text.
\usepackage{makecell} % Multiline table headings

% Allows clickable links for references, in table of content, autoref, etc.
\usepackage{hyperref}
\hypersetup{colorlinks, citecolor=black,
   		 	filecolor=black, linkcolor=black,
    		urlcolor=black}

%% Bibliography https://www.overleaf.com/learn/latex/Bibliography_management_with_biblatex
\usepackage[style=ieee]{biblatex} % style=apa also possible
\usepackage{csquotes}
\usepackage{listings}
\lstset{
basicstyle=\small\ttfamily,
columns=flexible,
breaklines=true,
numbers=left, numberstyle=\tiny,
showtabs=true
}
\addbibresource{references.bib}

% OPTIONAL SETTINGS (DELETE OR COMMENT TO SUPRESS)




% Define the number of section levels to be included in the t.o.c. and numbered	(3 is default)
\setcounter{tocdepth}{5}
\setcounter{secnumdepth}{5}


% Chapter title settings
\usepackage{titlesec}
\titleformat{\chapter}[display]
  {\Huge\bfseries\filcenter}
  {{\fontsize{50pt}{1em}\vspace{-4.2ex}\selectfont \textnormal{\thechapter}}}{1ex}{}[]


% Header and footer settings (Select TWOSIDE or ONESIDE layout below)
\usepackage{fancyhdr}
\pagestyle{fancy}
\renewcommand{\chaptermark}[1]{\markboth{\thechapter.\space#1}{}}


% Select one-sided (1) or two-sided (2) page numbering
\def\layout{2}	% Choose 1 for one-sided or 2 for two-sided layout
% Conditional expression based on the layout choice
\ifnum\layout=2	% Two-sided
    \fancyhf{}
	\fancyhead[LE,RO]{\nouppercase{ \leftmark}}
	\fancyfoot[LE,RO]{\thepage}
	\fancypagestyle{plain}{			% Redefine the plain page style
	\fancyhf{}
	\renewcommand{\headrulewidth}{0pt}
	\fancyfoot[LE,RO]{\thepage}}
\else			% One-sided
  	\fancyhf{}
	\fancyhead[C]{\nouppercase{ \leftmark}}
	\fancyfoot[C]{\thepage}
\fi


% Enable To-do notes
\usepackage[textsize=tiny,disable]{todonotes}   % Include the option "disable" to hide all notes
% \usepackage[textsize=tiny]{todonotes}   % Include the option "disable" to hide all notes
\setlength{\marginparwidth}{2.5cm}


% Supress warning from Texmaker about headheight
\setlength{\headheight}{15pt}



% Glossaries
\makeglossaries

\newglossaryentry{latex}
{
    name=latex,
    description={Is a markup language specially suited 
    for scientific documents}
}

\newglossaryentry{maths}
{
    name=mathematics,
    description={Mathematics is what mathematicians do}
}

