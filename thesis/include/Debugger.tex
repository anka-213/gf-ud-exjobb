\chapter{Debugger}

\section{Problem description}

An issue that was encountered when trying to write annotations for \texttt{ud2gf} is that sometimes it can be difficult to figure out why a rule does not fire.

The goal of the debugger project is to automatically give an explanation of why a rule (function label description) doesn't match.


% TODO: Better example

% Let's say we have the function $DetCN : Det -> CN -> NP ; det head$ and want to know why it wasn't used. We would call the tool with the argument \verb|dbg="DetCN 3 5"| and we would get out an explantation

Let's say we have the sentence ``colorless green ideas sleep furiously'' and believe that the function \verb|DetCN : Det -> CN -> NP ; det head| should be applied to ``green'' and ``ideas'' respectively, but it doesn't seem to be
used for some reason. Then we would call the \texttt{ud2gf} tool with either \verb|dbg="DetCN 2 3"| or \verb|dbg="DetCN green ideas"| and the debugger would give back one of a list of reasons to what prevented it from being applied:

\begin{itemize}
    \item The function has been manually disabled.
    \item The function does not exist.
    \item There are multiple definitions of the function.
    \item The wrong number of arguments was given to the debugger (meta-error).
    \item The requested indices are not found within the UD tree.
    \item The non-head arguments are not direct children of the head argument.
    \item The non-head arguments have labels that don't match the labels within the UD-tree.
    \item Required features are missing for some of the arguments.
    \item No trees with the required category could be found for some of the arguments.
    \item The function was possible to apply, but was pruned away because a different tree with the same category had higher priority.
\end{itemize}

If a word exist multiple times in the sentence, you need to use the number form, otherwise you will get an ``Ambiguous word'' error. In order to see the index number of a word you can either look at the conllu file\footnote{Conllu is the standard file format for UD trees.} directly or run \verb|ud2gf| with the \verb|ud| or \verb|ut| argument.


\section{Implementation}

The implementation is mostly a case of going through every step of the algorithm and seeing what can go wrong. Some of this process was simplified by the strong type system of Haskell, since it makes edge cases explicit, so in most cases we would get a type error if we missed handling a possible error.

In order to simplify the implementation and reduce code duplication, we first run the main algorithm to generate all possible GF trees in each node of the UD tree and use this result to check what goes wrong with the specific function we are interested in.

\section{Example}

Let's look at the phrase ``large portion of walls'', which in UD format looks like this:

\begin{center}
    \begin{dependency}
        \begin{deptext}[column sep=0.4cm]
              {\tt large}\&{\tt portion}\&{\tt of}\&{\tt walls}\\
            {\tt ADJ}\&{\tt NOUN}\&{\tt ADP}\&{\tt NOUN}\\
        \end{deptext}
        \depedge{2}{1}{amod}
        \depedge{2}{4}{nmod}
        \depedge{4}{3}{case}
        \deproot{2}{root}
    \end{dependency}
\end{center}
    
First we try to form a tree of words that are not in a head--dependent relation. The tool reports that \emph{large} is not a child of \emph{walls}, and suggests an actual child of \emph{walls} (i.e. \emph{of}) as a hint for the grammar writer.

\begin{lstlisting}
$ cat tests/examples/large_portion_of_walls.conllu | gf-ud ud2gf tests/grammars/Test Eng UDS no-backups dbg='AdjCN large walls'

Starting debug for AdjCN:
AdjCN : AP -> CN -> CN ; amod head
Error: Word number 1 ("large")  is not a child of 4 ("walls").
    Available children: [("3","of")]
\end{lstlisting}

Next, we ask the debugger about a correct attachment, but we try to apply a function of the wrong type. The debugger answers with ``Incompatible argument labels''.


\begin{lstlisting}
$ cat tests/examples/large_portion_of_walls.conllu | gf-ud ud2gf tests/grammars/Test Eng UDS no-backups dbg='DetCN large portion'

Starting debug for DetCN:
DetCN : Det -> CN -> NP ; det head
Attempting to build: DetCN large portion
Error: Incompatible argument labels:
 - For "large": Got amod expected det
\end{lstlisting}

Finally, if we debug a correct application, we get a step by step trace of how the tree is constructed.

\begin{lstlisting}
$ cat tests/examples/large_portion_of_walls.conllu | gf-ud ud2gf tests/grammars/Test Eng UDS no-backups dbg='AdjCN large portion'

Starting debug for AdjCN:
AdjCN : AP -> CN -> CN ; amod head
Attempting to build: AdjCN large portion

Argument "large" : AP ; amod:
  Found trees with correct category:
    - (PositA large_A) : AP

Argument "portion" : CN ; head:
  Found trees with correct category:
    - (AdvCN (UseN portion_N) (PrepNP of_Prep (DetCN_aPl (UseN wall_N)))) : CN
Trees using AdjCN found in devtree:
    (AdjCN (PositA large_A) (AdvCN (UseN portion_N) (PrepNP of_Prep (DetCN_aPl (UseN wall_N))))) : CN
Success!
\end{lstlisting}


% In this case the tree was successfully built, but pruned away because other trees had a higher priority. This typically means that you need to disable some functions and replace them with more specific functions to ensure that the desired functions are used.

% First the main algorithm is run and afterwards all the generated trees are searched, this can lead to some information being missing compared to actually tracing running the algorithm and detecting where where it goes wrong. However in practice it works well enough and it significantly simplifies the implementation, since the main algorithm doesn't need to be modified nor duplicated in order to implement the debugger.
