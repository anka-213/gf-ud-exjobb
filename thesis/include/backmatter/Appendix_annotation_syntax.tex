\chapter{Annotation syntax}\label{app:annotation}

Annotations are required in order for gf-ud to know how the GF categories and functions relate to the UD Part of Speech tags and dependency labels. There are two main categories of annotations: Abstract annotations that are associated with a GF abstract syntax and are language independent and Concrete annotations that are associated with a GF concrete syntax and are language specific.

The following appendix is based on the original documentation\footnote{\url{https://github.com/GrammaticalFramework/gf-ud/blob/master/doc/annotations.md}} written by Aarne Ranta and updated to include new syntax. 

\hypertarget{abstract-annotations}{%
\section{Abstract annotations}\label{abstract-annotations}}

These annotations are associated with an abstract syntax and are in general not specific for a concrete language, but in some cases additional abstract annotations may be needed for a specific language.

%All of these annotations are bidirectional and language-independent, since they only refer to the abstract syntax in GF.

\textbf{Function}: \texttt{\#fun}

The \lstinline{#fun} annotation describes GF functions and which UD dependency label each argument of the function should have (see \autoref{fig:DetCN} for graphical representation of such an annotation). The type signature of the GF function is also required. Each function needs to have at least one head with the label \lstinline{head} and any number of children which are labeled with their corresponding UD dependency labels. 

\begin{lstlisting}
    #fun GFFunctionName : FirstArgumentCat -> SecondArgumentCat -> ReturnCat ; first_ud_label second_ud_label
    #fun UseN : N -> CN ; head
    #fun DetCN  : Det -> CN -> NP ; det  head    
\end{lstlisting}

The function \lstinline{UseN} only has a single argument (of type \lstinline{N}), so that is required to be the head, while the function \verb|DetCN| has two arguments, of types \verb|Det| and \verb|CN|, the first of which has the dependency label \verb|det|, while the second is the head.

It is also possible to match the arguments by their inflectional features. The syntax for that is to put the feature and its value in square brackets, attached to a label. Subtypes for the relations, such as \texttt{nsubj:pass}, can be pattern matched with a wildcard \verb|*|.

\begin{lstlisting}
    #fun SingularNP : Det -> CN -> NP ; det head[Number=Sg]
    #fun AnySubject : NP -> VP -> Cl ; nsubj:* head
\end{lstlisting}


\textbf{Category}: \texttt{\#cat}

The \lstinline{#cat} annotation describes the mapping from GF categories to UD POS tags.

\begin{lstlisting}
    #cat GFCategory UD_POS
    #cat N NOUN
\end{lstlisting}

Many categories can have the same POS tag. Only those categories that
have lexical items (i.e.~zero-place functions linearized to words) need
this annotation.

Both function and category annotations must be unique in order for
\texttt{gf2ud} to work. 
However, in the \texttt{ud2gf} direction, two kinds of deviations may be needed: \texttt{\#auxcat} and \texttt{\#auxfun}, which are presented in Section~\ref{special-concrete-annotations-for-ud2gf}.

%%%


\textbf{Annotation guideline}: \texttt{\#guidelines}

\begin{verbatim}
  #guidelines UD2
\end{verbatim}

%This is only used when checking the validity of an annotation file. The two valid options are \verb|#guidelines UD2| or \verb|#guidelines none|. If the option is omitted, UD2 is assumed. The option UD2 means that only valid UD labels are allowed, while none means that no such checks are performed.

This flag is used to indicate what guidelines the annotations are
checked against (if any). The default it UD2. It means checking that
each label and POS tag is an actually existing UD2 label or tag. An
alternative is \texttt{none}, meaning that no checks are performed.



\hypertarget{concrete-annotations}{%
\section{Concrete annotations}\label{concrete-annotations}}

\textbf{Syncategorematic words}: \texttt{\#word}

\begin{verbatim}
  #word is  be Mood=Ind|Number=Sing|Person=3|Tense=Pres|VerbForm=Fin
\end{verbatim}

These annotations are used for words that appear syncategorematically,
i.e.~elsewhere than as linearizations of zero-place functions. The lemma
should be defined in a \texttt{\#lemma} annotation

\textbf{Syncategorematic lemmas}: \texttt{\#lemma}

\begin{verbatim}
  #lemma UseAP,UseAdv,UseNP be Cop cop head
\end{verbatim}

This means that when any of the functions \texttt{UseAP,\ UseAdv,\ UseN}
introduces the lemma \texttt{be}, it is to be marked with the category
\texttt{Cop} and the label \texttt{cop}. Its head will be the head of
the same construction. Notice that the category is typically an
auxiliary category (see \texttt{\#auxcat} below). A shortcut exists for
saying that a lemma has the same properties in all functions, except
those explicitly defined:

\begin{verbatim}
  #lemma DEFAULT_ be Cop cop head
\end{verbatim}

\textbf{Morphological tags}: \texttt{\#morpho}

\begin{verbatim}
  #morpho V,V2 2 Mood=Ind|Tense=Past|VerbForm=Fin
\end{verbatim}

This annotations tells how the PGF coordinates (numbers) are converted
to morphological tags of UD. The conversion depends on the category,
since different categories have different PGF layouts. To find out what
the coordinates mean, one can use \texttt{gfud\ check-annotations},
which returns examples such as

\begin{verbatim}
  morpho mapping missing:
  #morpho N 1 -- s Sg Gen A-bomb's
  #morpho N 3 -- s Pl Gen A-bombs'
\end{verbatim}

This means that all fields except 1 and 3 have been defined for the
category N. One can also use the text dump of the PGF invoked by the
\texttt{print\_grammar} command of GF:

\begin{verbatim}
  > import ShallowParse.pgf
  > print_grammar
    V := range  [C32 .. C32]
         labels ["s Inf"
                 "s PresSg3"
                 "s Past"
                 "s PastPart"
                 "s PresPart"]
\end{verbatim}

The numbering of labels starts from 0. Thus the third label,
\texttt{s\ Past}, is the one with number 2. Comparing with a UD2
treebank shows that the morphological tag to be used is
\texttt{Mood=Ind\textbar{}Tense=Past\textbar{}VerbForm=Fin}.

\textbf{Discontinuous constituents}: \texttt{\#discont ,head}

\begin{verbatim}
  #discont V2 0-4,head   5,ADV,advmod,head  6,ADP,case,obj
\end{verbatim}

This is applied to discontinuous words, such as V2 with its verb
particle (field 5) and preposition (field 6). The head fields are those
that contain actual verb forms. The other fields need a POS tag, the
word's own label, and the label of the word in the construction that
becomes the head. In this example, the particle (5) is linked to the
head of the construction (i.e.~the verb), whereas the preposition (6) is
linked to the object.

\emph{NB: the target label functionality is currently not implemented, but
all fields are linked to the head.}

\textbf{Multiwords}:
\texttt{\#multiword\ \textless{}category\textgreater{}\ \textless{}head-position\textgreater{}\ \textless{}label\textgreater{}}:

\begin{verbatim}
  #multiword Prep head-first fixed
\end{verbatim}

This annotation tells that, in a multiword preposition, the first word
is the head, and the subsequent words are its dependents with the label
\texttt{fixed}. This follows the guidelines in
https://universaldependencies.org/u/overview/specific-syntax.html;
however, existing treebanks don't seem to follow this strictly. The
guideline example is \emph{in spite of}, where \emph{in} is the \texttt{head}
and the other words are \texttt{fixed}.

\emph{NB: \#multiword works so far only in the gf2ude direction, and
head-last is not supported at all yet}

\textbf{Change of label}:
\texttt{\#change\ \textless{}label\textgreater{}\ \textgreater{}\ \textless{}label\textgreater{}\ \textless{}condition\textgreater{}}:

\begin{verbatim}
  #change obj > obl above case
  #change det > nmod:poss features Poss=Yes|PronType=Prs
\end{verbatim}

These annotations are used at the last step of gf2ud to capture
discrepancies between the grammar and the annotation standard that are
difficult to define in another way. The \texttt{above} condition changes
the label in a node that dominates a certain label. In the example
shown, the \texttt{case} label appears in object that has a preposition,
which is a concrete syntax feature. The \texttt{features} condition
changes the label if the node has certain morphological features. In the
example, possessive pronouns are get the label \texttt{nmod:poss}
instead of the usual \texttt{det}.

\hypertarget{special-concrete-annotations-for-ud2gf}{%
\subsection{Special concrete annotations for
ud2gf}\label{special-concrete-annotations-for-ud2gf}}

The previously shown annotations are used in both directions, gf2ud and
ud2gf. The ud2gf direction uses some additional annotations, listed
here:

\textbf{Alternative function}: \texttt{\#altfun}

\begin{verbatim}
  #altfun ComplV2 head obl
\end{verbatim}

This is needed in \texttt{ud2gf} for reading normal UD, because the
complement of a V2 verb can be labelled either \texttt{obj} of
\texttt{obl} depending on the case governed by the verb.

\textbf{Disabled function}: \texttt{\#disable}

\begin{verbatim}
  #disable UseAP
\end{verbatim}

says that the function \texttt{UseAP} is not to be used in
\texttt{ud2gf}. The reason is that it is overshadowed by a macro
function introducing an auxiliary copula.

\textbf{Auxiliary category}: \verb|#auxcat <abstract syntax category> <POS tag>|

\begin{verbatim}
   #auxcat Cop AUX
\end{verbatim}

The category \texttt{Cop} is introduced in the concrete syntax
annotation for the syncategorematic copula verb (see below). In the
\texttt{ud2gf} direction, it is recognized when such a verb, marked by
the POS tag \texttt{AUX}, is encountered in the dependency tree. This is
done by means of auxiliary functions, whose format is somewhat complex,
as they also have to define these extra functions in terms of standard
ones.

\textbf{Auxiliary function}: \texttt{\#auxfun}


\begin{lstlisting}
 #auxfun <new abstract syntax function> <argument variables> : <type> = <definition> ; <labels-and-features>|

 #auxfun MkVPS_Perf have vp : Have -> VP -> VPS = MkVPS (TTAnt AAnter TPres) PPos vp ; aux[Tense=Pres] head
\end{lstlisting}

The number of argument variables and labels must match the type. The
type can be built from both ordinary and auxiliary categories. As the
definition shows here, the auxiliary category \texttt{Have} argument is
ignored. But the important thing is that - its presence in the UD tree
is recognized, ensuring that all words are taken into account. - it sets
the tense and anteriority of the VPS formed

Auxiliary functions do not necessarily contain auxiliary categories:
they can be just macros collecting the applications of many functions.



\textbf{Recursive macros}

\emph{New with this work:} Auxiliary functions can also be defined in terms of other auxfuns, in which case they will be recursively evaluated. This is useful for transformations that require a larger context than the immediate children of the tree.

The following helper macros can be used for capturing multiple nodes in a pair that can then later be deconstructed somewhere higher up in the tree:

\begin{lstlisting}
-- ** Generic, only used inside other macros **
-- Pair_a_b = (a -> b -> r) -> r
#auxfun MkPair_ a b handler : a -> b -> ab2r -> r = handler a b ; head dummy nonexistent
#auxfun UsePair_ handler pair : ab2r -> Pair_a_b -> r = pair handler ; head dummy
-- Triple_a_b_c = (a -> b -> r) -> r
#auxfun MkTriple_ a b c handler : a -> b -> c -> abc2r -> r = handler a b c ; head dummy nonexistent nope
#auxfun UseTriple_ handler triple : abc2r -> Triple_a_b_c -> r = triple handler ; head dummy
\end{lstlisting}

When consuming a pair, you'll typically need a second helper macro (in this example \texttt{AP\_helper\_}), because there are no closures in this simple expansion, everything.
\begin{lstlisting}
#auxfun AndCuteCont_ and cute : Conj -> AP -> Pair_Conj_AP = MkPair_ and cute ; cc head

#auxfun AP2_ small andCute : AP -> Pair_Conj_AP -> AP = UsePair_ (AP2_helper_ small) andCute ; head conj
#auxfun AP2_helper_ small and cute :  AP -> Conj -> AP -> AP = ConjAP and (BaseAP small cute) ; head dummy nonexistent
\end{lstlisting}

The helper macros that are only intended for use inside other macros have dummy labels and dummy types, since they are not intended to be matched by the main algorithm.