% CREATED BY DAVID FRISK, 2016
\chapter{Introduction}
This chapter presents the section levels that can be used in the template. 

\section{Section levels}
\autoref{tab:sections} presents an overview of the section levels that are used in this document. The number of levels that are numbered and included in the table of contents is set in the settings file \texttt{Settings.tex}. The levels are shown in Section \ref{Section_ref}.

This is a new paragraph and should have proper parskip or indentation. Don't forget to cite your sources~\cite{Brajnik2008}. % '~' becomes space which cannot line break.

\begin{table}[h]
\centering
\caption{Section levels} % Table text above table.
\begin{tabular}{ll} \hline
Name & Command\\ \hline
Chapter & \textbackslash\texttt{chapter\{\emph{Chapter name}\}}\\
Section & \textbackslash\texttt{section\{\emph{Section name}\}}\\
Subsection & \textbackslash\texttt{subsection\{\emph{Subsection name}\}}\\
Subsubsection & \textbackslash\texttt{subsubsection\{\emph{Subsubsection name}\}}\\
%Paragraph & \textbackslash\texttt{paragraph\{\emph{Paragraph name}\}}\\
%Subparagraph & \textbackslash\texttt{paragraph\{\emph{Subparagraph name}\}}\\ \hline\hline
\end{tabular}
\label{tab:sections}
\end{table}

\section{Section} \label{Section_ref}
\subsection{Subsection}
\subsubsection{Subsubsection}
%\paragraph{Paragraph}
%\subparagraph{Subparagraph}

1. Intro
- dep trees,
 UD
 - abstr trees, GF
 - strengths and weaknesses
  -> useful synthesis: UD-GF
  -> applications:
    - translation  - CL 2020
    - semantics    - CL 2020
    - concept alignment - Arianna
    - SMU work - analysis of real world law text
